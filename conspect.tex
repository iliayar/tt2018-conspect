\documentclass[12pt, a4paper]{article}
\usepackage{graphicx}
\usepackage{stmaryrd}
\usepackage[T1,T2A]{fontenc}
\usepackage[utf8]{inputenc}
\usepackage[english,russian]{babel}
\usepackage{amsmath}
\usepackage{amsfonts}
\usepackage{amssymb}
\usepackage{makeidx}
\usepackage{verbatim}
\usepackage{amsthm}
\usepackage{bnf}
\usepackage{tikz}
\usepackage{enumerate}
\usepackage{mathtext}
\usepackage{mathtools}
\usepackage{mathabx}
\usepackage[left=2cm,right=2cm,top=2cm,bottom=2cm,bindingoffset=0cm]{geometry}
\usepackage{proof}
\usepackage{paracol}
\usepackage{enumitem}
\usepackage{color}
\usepackage{colortbl}
\usepackage{minted}
\usetikzlibrary{graphs}
\usetikzlibrary{graphs.standard}
\usetikzlibrary{automata,positioning}
\usepackage{float}

\title{Введение в Теорию Типов\\\it{Конспект лекций}}
\author{Штукенберг Д.~Г.\\Университет ИТМО}
\DeclareMathOperator{\FV}{FV}

\begin{document}

\theoremstyle{definition}
\newtheorem{definition}{Определение}[section]
\newtheorem{note}{Замечание}[section]

\newtheorem{axiom}{Аксиома}[section]
\newtheorem{theorem}{Теорема}[section]
\newtheorem{lemma}[theorem]{Лемма}
\newtheorem{statement}{Утверждение}[section]
\newtheorem{oun_paragraph}{Пункт}[section]
\newtheorem{cons}{Следствие}[section]
\newtheorem*{example}{Пример}

\newcommand{\comb}[1]{\operatorname{\mathcal{#1}}}
\newcommand{\func}[1]{\operatorname{#1}}
\newcommand{\reduction}[1]{{\color{OrangeRed}#1}}
\newcommand{\set}[1]{\left\{#1\right\}}

\def\from#1{\par \parbox{0.7\textwidth}{\par \hfill\raggedleft \it #1}} 


\newenvironment{epigraph}% 
{\begin{list}{}{\setlength{\leftmargin}{0.3\textwidth}}\item[]}% 
{\end{list}}

\maketitle
\tableofcontents
\newpage

\input{lection01}
\input{lection02}
\section{Лекция 3}

\subsection{Y-комбинатор}

\begin{definition}
	Комбинатором называется $\lambda$-выражение, не имеющее свободных переменных
\end{definition}

\begin{definition}($Y$-комбинатор)
	\[
	Y = \lambda f . (\lambda x . f (x x)) (\lambda x . f (x x))
	\]
\end{definition}

Очевидно, $Y$-комбинатор является комбинатором.

\begin{theorem}
	$Y f =_{\beta} f (Y f)$
	
	\begin{proof}
		$\beta$-редуцируем выражение $Y f$
		\begin{align*}
			 =_{\beta} \textcolor{magenta}{(\lambda f . (\lambda x . f (x x)) (\lambda x . f (x x)))}\textcolor{blue}{f} \\ =_{\beta} \textcolor{magenta}{(\lambda x . f (x x))} \textcolor{blue}{(\lambda x . f (x x))} \\ =_{\beta} f ((\lambda x . f (x x))(\lambda x . f (x x))) \\ =_{\beta} f(Y f)
		\end{align*}
		Так как при второй редукции мы получили, что $Y f =_{\beta} (\lambda x . f (x x))(\lambda x . f (x x))$
	\end{proof}	
\end{theorem}

Следствием этого утверждения является теорема о неподвижной точке для бестипового $\lambda$-исчисления

\begin{theorem}
	В $\lambda$-исчислении каждый терм $f$ имеет неподвижную точку, то есть такое $p$, что $f \; p =_{\beta} p$
	
	\begin{proof}
		Возьмём в качестве $p$ терм $Y f$. По предыдущей теореме, $f(Y f) =_{\beta} Y f$, то есть $Y f$ является неподвижной точкой для $f$. Для любого терма $f$ существует терм $Y f$, значит, у любого терма есть неподвижная точка.
	\end{proof}
\end{theorem}

\subsection{Рекурсия}

С помощью $Y$-комбинатора можно определять рекурсивные функции, например, функцию, вычисляющую факториал Чёрчевского нумерала. Для этого определим вспомогательную функцию

$fact' \equiv \lambda f. \lambda n. isZero\; n \; \overline{1} (mul \; n \; f((-1) n))$

Тогда $fact \equiv Y fact'$

Заметим, что $fact \; \overline{n} =_{\beta} fact' \; (Y \; fact') \; \overline{n} =_{\beta}fact' \; fact \; \overline{n} $, то есть в тело функции $fact'$ вместо функции $f$ будет подставлена $fact$ (заметим, что это значит, что именно функция $fact$ будет применена к $\overline{n - 1}$, то есть это соответствует нашим представлениям о рекурсии).

Для понимания того, как это работает, посчитаем $fact \; \overline{2}$

\begin{align*}
	fact \; \overline{2} \\ =_{\beta} Y \; fact' \; \overline{2}\\ =_{\beta} fact' (Y \; fact') \overline{2} \\=_{\beta}\textcolor{magenta}{(\lambda f. \lambda n. isZero\; n \; \overline{1} (mul \; n \; f((-1) n)))} \textcolor{blue}{(Y \; fact') \overline{2}} \\
	=_{\beta}isZero\; \overline{2} \; \overline{1} (mul \; \overline{2} \; ((Y \; fact')((-1) \overline{2}))) \\ =_{\beta} mul \; \overline{2} \; ((Y \; fact')((-1) \overline{2})) \\ =_{\beta} mul \; \overline{2} \; (Y \; fact' \; \overline{1}) \\ =_{\beta} mul \; \overline{2} \; (fact' \;(Y \; fact' \; \overline{1}))
\end{align*}

Раскрывая $fact' \;(Y \; fact' \; \overline{1})$ так же, как мы раскрывали  $fact' \;(Y \; fact' \; \overline{2})$, получаем

\begin{align*}
	=_{\beta} mul \; \overline{2} \; (mul \; \overline{1} \; (Y \; fact' \; \overline{0}))
\end{align*}

Посчитаем $(Y \; fact' \; \overline{0})$

\begin{align*}
	(Y \; fact' \; \overline{0}) \\ =_{\beta} fact' \; (Y \; fact') \; \overline{0} \\ =_{\beta} \textcolor{magenta}{(\lambda f. \lambda n. isZero\; n \; \overline{1} (mul \; n \; f((-1) n)))} \; \textcolor{blue}{(Y \; fact') \; \overline{0}} \\ =_{\beta}  isZero\; \overline{0} \; \overline{1} (mul \; \overline{0} \; ((Y \; fact'))((-1) \overline{0})) =_{\beta} \overline{1}
\end{align*}

Таким образом,

\begin{align*}
	fact \; \overline{2} \\ =_{\beta} mul \; \overline{2} \; (mul \; \overline{1} \; (Y \; fact' \; \overline{0})) \\=_{\beta} mul \; \overline{2} \; (mul \; \overline{1} \; \overline{1}) =_{\beta} mul \; \overline{2} \; \overline{1} =_{\beta} \overline{2}
\end{align*}

\subsection{Парадокс Карри}

Попробуем построить логику на основе $\lambda$-исчисления. Введём логический символ $\rightarrow$. 

Будем требовать от этого исчисления наличия следующих схем аксиом:

\begin{enumerate}
	\item $\vdash A \rightarrow A$
	\item $\vdash (A \rightarrow (A \rightarrow B)) \rightarrow (A \rightarrow B)$
	\item $\vdash A =_{\beta} B$, тогда $A \rightarrow B$
\end{enumerate}

А также правила вывода MP:

$$\infer{\vdash B}{\vdash A \rightarrow B && \vdash A}$$

Не вводя дополнительные правила вывода и схемы аксиом, покажем, что данная логика является противоречивой. Для чего введём следующие условные обозначения:

$F_{\alpha} \equiv \lambda x. (x \; x) \rightarrow \alpha$

$\Phi_{\alpha} \equiv F_{\alpha}  \; F_{\alpha}  \equiv (\lambda x. (x \; x) \rightarrow \alpha) \; (\lambda x. (x \; x) \rightarrow \alpha)$

Редуцируя $\Phi_{\alpha}$, получаем 

\begin{align*}
\Phi_{\alpha} \\ =_{\beta} \textcolor{magenta}{(\lambda x. (x \; x) \rightarrow \alpha)} \; \textcolor{blue}{(\lambda x. (x \; x) \rightarrow \alpha)} \\=_{\beta} (\lambda x. (x \; x) \rightarrow \alpha) \; (\lambda x. (x \; x) \rightarrow \alpha) \rightarrow \alpha \\=_{\beta} \Phi_{\alpha} \rightarrow \alpha
\end{align*}

Теперь докажем противоречивость введённой логики. Для этого докажем, что в ней выводимо любое утверждение.

\begin{tabular}{ll}
	1) $\vdash\Phi_\alpha\rightarrow\Phi_\alpha\rightarrow\alpha$ & Так как $\Phi_{\alpha} =_{\beta} \Phi_{\alpha} \rightarrow \alpha$\\
	2) $\vdash(\Phi_\alpha\rightarrow\Phi_\alpha\rightarrow\alpha)\rightarrow(\Phi_\alpha\rightarrow\alpha)$ & Так как $\vdash (A \rightarrow (A \rightarrow B)) \rightarrow (A \rightarrow B)$\\
	3) $\vdash\Phi_\alpha\rightarrow\alpha$ & MP 1, 2\\
	4) $\vdash (\Phi_\alpha \rightarrow \alpha) \rightarrow \Phi_\alpha$ & Так как $\vdash \Phi_\alpha \rightarrow \alpha =_{\beta} \Phi_\alpha$\\
	5) $\vdash\Phi_\alpha$ & MP 3, 4\\
	6) $\vdash\alpha$ & MP 3, 5
\end{tabular}

Таким образом, введённая логика оказывается противоречивой.

\subsection{Импликационный фрагмент интуиционистского исчисления \\высказываний}

Рассмотрим подмножество ИИВ, со следующей грамматикой:

$\Phi ::= x \; | \; \Phi \rightarrow \Phi \; | \; (\Phi)$

То есть состоящее только из переменных и импликаций. 

Добавим в него одну схему аксиом

$$\Gamma, \varphi \vdash \varphi$$

И два правила вывода

\begin{enumerate}
	\item Правило введения импликации:
	\[
	\infer{\Gamma \vdash \varphi \to \psi}{\Gamma, \varphi \vdash \psi}
	\]
	\item Правило удаления импликации:
	\[
	\infer{\Gamma \vdash \psi}{\Gamma \vdash \varphi \to \psi && \Gamma \vdash \varphi}
	\]
\end{enumerate}

\begin{example}
	Докажем $\vdash \varphi \rightarrow \psi \rightarrow \varphi$
	
	\[
	\infer[(\text{Введение импликации})]
	{ \vdash \varphi \to (\psi \to \varphi) }
	{ \infer[(\text{Введение импликации})]
		{ \varphi \vdash \psi \to \varphi }
		{\varphi, \psi \vdash \varphi}
	}
	\]
\end{example}

\begin{example}
	Докажем $\alpha \rightarrow \beta \rightarrow \gamma, \; \alpha, \; \beta \vdash \gamma$
	
	\[
	\infer
	{ \alpha \rightarrow \beta \rightarrow \gamma, \; \alpha, \; \beta \vdash \gamma}{\infer
		{\alpha \rightarrow \beta \rightarrow \gamma, \; \alpha, \; \beta \vdash \beta \rightarrow \gamma }{\alpha \rightarrow \beta \rightarrow \gamma, \; \alpha, \; \beta \vdash \alpha \rightarrow \beta \rightarrow \gamma && \alpha \rightarrow \beta \rightarrow \gamma, \; \alpha, \; \beta \vdash \alpha} && \alpha \rightarrow \beta \rightarrow \gamma, \alpha, \; \beta \vdash \beta}
	\]
	
\end{example}

\subsection{Просто типизированное по Карри $\lambda$-исчисление}

\begin{definition}
	Тип в просто типизированном $\lambda$-исчислении по Карри --- это либо маленькая греческая буква ($\alpha, \phi, \theta, \ldots$), либо импликация ($\theta_1 \rightarrow \theta_2$)
	
	Таким образом, $\Theta ::= \theta_{i} | \Theta \rightarrow \Theta | (\Theta)$
	
	Импликация при этом считается правоассоциативной операцией.
\end{definition}

\begin{definition}
	Язык просто типизированного $\lambda$-исчисления --- это язык бестипового $\lambda$-исчисления.
\end{definition}

\begin{definition}
	Контекст $\Gamma$ --- это список выражений вида $A: \theta$, где $A$ --- $\lambda$-терм, а $\theta$ --- тип.
\end{definition}

\begin{definition}
	Просто типизированное $\lambda$-исчисление по Карри.
	
	Рассмотрим исчисление с единственной схемой аксиом:
	
	$$\Gamma, x : \theta \vdash x : \theta, \text{если } x \text{ не входит в } \Gamma$$
	
	И следующими правилами вывода
	
	\begin{enumerate}
		\item Правило типизации абстракции
		\[
		\infer[\text{если } x \text{ не входит в } \Gamma]{\Gamma \vdash (\lambda \; x. \; P) : \varphi \rightarrow \psi}{\Gamma, x : \varphi \vdash P : \psi}
		\]
		\item Правило типизации аппликации:
		\[
		\infer{\Gamma \vdash PQ : \psi}{\Gamma \vdash P : \varphi \to \psi && \Gamma \vdash Q : \varphi}
		\]
	\end{enumerate}

	Если $\lambda$-выражение типизируется с использованием этих двух правил и одной схемы аксиом, то будем говорить, что оно типизируется по Карри.
\end{definition}

\begin{example}
	Докажем $\vdash \lambda \; x. \; \lambda \; y. \; x : \alpha \rightarrow \beta \rightarrow \alpha$
	
	\[
	\infer[(\text{Правило типизации абстракции})]
	{ \vdash \lambda \; x. \; \lambda \; y. \; x : \alpha \rightarrow \beta \rightarrow \alpha }
	{ \infer[(\text{Правило типизации абстракции})]
		{x: \alpha \vdash \lambda \; y. \; x : \beta \rightarrow \alpha}
		{x: \alpha, y : \beta \vdash x : \alpha}
	}
	\]
\end{example}


\begin{example}
	Докажем $\vdash \lambda \; x. \; \lambda \; y. \; x \; y : (\alpha \rightarrow \beta) \rightarrow \alpha \rightarrow \beta$
	
	\[
	\infer
	{\vdash \lambda \; x. \; \lambda \; y. \; x \; y : (\alpha \rightarrow \beta) \rightarrow \alpha \rightarrow \beta}
	{
		\infer
		{x: \alpha \rightarrow \beta \vdash \lambda \; y. \; x \; y : \alpha \rightarrow \beta}
		{
			\infer
			{x: \alpha \rightarrow \beta, y : \alpha \vdash x \; y : \beta}
			{x: \alpha \rightarrow \beta, y : \alpha \vdash x: \alpha \rightarrow \beta && x: \alpha \rightarrow \beta, y : \alpha \vdash y: \alpha}
		}
	}
	\]
\end{example}

\subsection{Отсутствие типа у Y-комбинатора}

\begin{theorem}
	$Y$-комбинатор не типизируется в просто типизированном по Карри $\lambda$-исчислении.
\end{theorem}

\subparagraph{Неформальное доказательство}

$Y \; f =_{\beta} f \; (Y \; f)$, поэтому $Y \; f$ и $f \; (Y \; f)$ должны иметь одинаковые типы.

Пусть $Y \; f : \alpha$

Тогда $Y : \beta \rightarrow \alpha, f : \beta$

Из $f \; (Y \; f) : \alpha$ получаем $f: a \rightarrow \alpha$ (так как $Y f : \alpha$)

Тогда $\beta = \alpha \rightarrow \alpha$, из этого получаем $Y : (\alpha \rightarrow \alpha) \rightarrow \alpha$

Можно доказать, что $\lambda \; x. \; x : \alpha \rightarrow \alpha$. Тогда $Y \; \lambda \; x. \; x : \alpha$, то есть любой тип является обитаемым. Так как это невозможно, $Y$-комбинатор не может иметь типа, так как тогда он сделает нашу логику противоречивой.

\subparagraph{Формальное доказательство}

Докажем от противного. Пусть $Y$-комбинатор типизируем. Тогда в выводе его типа есть вывод типа выражения $x \; x$. Так как $x \; x$ --- абстракция, то и типизирована она может быть только по правилу абстракции. Значит, в выводе типа $Y$-комбинатора есть такой вывод:

$$\infer{\Gamma \vdash x x: \psi}{\Gamma \vdash x: \varphi \rightarrow \psi && \Gamma \vdash x: \varphi }$$

Рассмотрим типизацию $\Gamma \vdash x: \varphi \rightarrow \psi$ и $\Gamma \vdash x: \varphi$. $x$ это атомарная переменная, значит, она могла быть типизирована только по единственной схеме аксиом. 

Следовательно, $x$ типизируется следующим образом.

$$\infer{\Gamma', x: \varphi \rightarrow \psi, x: \varphi \vdash x x: \psi}{\Gamma', x: \varphi \rightarrow \psi, x: \varphi \vdash x: \varphi \rightarrow \psi && \Gamma', x: \varphi \rightarrow \psi, x: \varphi \vdash x: \varphi }$$

Следовательно, в контексте $\Gamma$ переменная $x$ встречается два раза, что невозможно по схеме аксиом.

\subsection{Изоморфизм Карри-Ховарда}

Заметим, что аксиомы и правила вывода импликационного фрагмента ИИВ и просто типизированного по Карри $\lambda$-исчисления точно соответствуют друг другу. 

$\newline$
\begin{tabular}{ | p{8cm} | p{8cm} | }
	\hline
	Просто типизированное $\lambda$-исчисление & Импликативный фрагмент ИИВ \\ \hline
	$\Gamma, x : \theta \vdash x : \theta$ & $\Gamma, \varphi \vdash \varphi$ \\
	&\\
	$\infer{\Gamma \vdash (\lambda \; x. \; P) : \varphi \rightarrow \psi}{\Gamma, x : \varphi \vdash P : \psi}$ & $\infer{\Gamma \vdash \varphi \to \psi}{\Gamma, \varphi \vdash \psi}$  \\
	&\\
	$\infer{\Gamma \vdash PQ : \psi}{\Gamma \vdash P : \varphi \to \psi && \Gamma \vdash Q : \varphi}$ & $\infer{\Gamma \vdash \psi}{\Gamma \vdash \varphi \to \psi && \Gamma \vdash \varphi}$ \\
	\hline
\end{tabular}

$\newline$
Установим соответствие и между прочими сущностями ИИВ и просто типизированного по Карри $\lambda$-исчисления.

$\newline$
\begin{tabular}{ | p{8cm} | p{8cm} | }
	\hline
	Просто типизированное $\lambda$-исчисление & Импликативный фрагмент ИИВ \\ \hline
	Тип & Высказывание \\
	Терм & Доказательство высказывания  \\
	Проверка того, что терм имеет заданный тип & Проверка доказательства на корректность \\
	Обитаемый тип & Доказуемое высказывание \\
	Проверка того, что существует терм, имеющий заданный тип & Проверка того, что заданное высказывание имеет доказательство \\
	\hline
\end{tabular}

\input{lection04}
\input{lection05}
		\section{Лекция 6 \\ Реконструкция типов в просто типизированном $\lambda$-исчислении, комбинаторы}
		\subsection{Алгоритм вывода типов}
					
			Пусть есть: $?\:\vdash \:A\: :\: ?$, хотим найти пару $\big \langle \text{контекст}, \text{тип} \big \rangle$\par
	\textbf{Алгоритм:}
	\begin{enumerate}
		\item Рекурсия по структуре формулы\par Построить по формуле $A$ пару $\big \langle E, \tau\big \rangle$, где\par $E-$система уравнений, $\tau-$тип $A$
		\item Решение уравнения, получение подстановки $S$ и из решения $E$ и $S\:(\tau)$ получение ответа	
	\end{enumerate}
		Т.е. необходимо свести вывод типа к алгоритму унификации.\par
		\begin{oun_paragraph}Рассмотрим 3 случая\end{oun_paragraph}
			\textbf{Обозначение } $\rightarrow-$ алгебраический тип 
			\begin{enumerate}
				\item $A\equiv x\implies\:\big \langle \{\}, \alpha_A\big\rangle$, где $\{\}-$пустой контекст, $\alpha_A-$новая переменная, нигде не встречавшаяся до этого в формуле
				\item $A\equiv P\:Q\implies\big \langle E_P\cup E_Q\cup \{\tau_P=\:\rightarrow\:(\tau_Q\:\alpha_A)\}, \alpha_A\big \rangle$, где $\alpha_A-$новая переменная
				\item $A\equiv\lambda x.P\implies\big\langle E_P,\alpha_x\:\rightarrow\:\tau_P\big\rangle$
			\end{enumerate}
		\begin{oun_paragraph}Алгоритм унификации\end{oun_paragraph} 
			Рассмотрим $E-$систему уравнений, запишем все уравнения в алгебраическом виде, т.е. \par $\alpha\:\rightarrow\:\beta\:\Leftrightarrow\:\rightarrow\:\alpha\:\beta$, затем применяем алгоритм унификации.
	\begin{lemma}
	Рассмотрим терм $M$ и пару $\big\langle E_M, \tau_M\big\rangle$, Если $\Gamma\:\vdash M:\rho$, то существует:
	\end{lemma}	
		\begin{enumerate}
		\item $S-$решение $E_M$ тогда $\Gamma=\{x : S(\alpha_x)\:|\:x\in FV(M)\}$, $FV-$множество свободных переменных в терме $M$, $\alpha_x-$ переменная, полученная при разборе терма $M$\par
		$\rho\:=\:S(\tau_M)$
		\item Если $S-$ решение $E_M$, то $\Gamma\:\vdash M:\rho$,
		\begin{proof}индукция по структуре терма $M$

			\begin{enumerate}
				\item Если $M\equiv x$, то так как решение существует, то существует и $S(\alpha_x)$, что: \par$\Gamma,\:x:S(\alpha_x)\:\vdash \:x:S(\alpha_x)$
				\item Если $M\equiv \lambda x.\:P$, то по индукции уже известен тип $P$, контекст $\Gamma$ и тип $x$, тогда: $${\Gamma,\:x:\:S(\alpha_x)\:\vdash \:P:\:S(\alpha_P) \over \Gamma\:\vdash \:\lambda x.\:P:S(\alpha_x)\:\rightarrow\:S(\alpha_P)}$$
				\item Если $M\equiv P\:Q$, то по индукции: $${ \Gamma\:\vdash \:P:\:S(\alpha_P)\equiv\tau_1\:\rightarrow\:\tau_2 \hspace{4em} \Gamma\:\vdash \:Q:S(\alpha_Q)\equiv\tau_1\over \Gamma\:\vdash \:P\:Q:\tau_2}$$
			\end{enumerate}					
		\end{proof}
	\end{enumerate}				
		 $\big \langle\Gamma,\rho\big\rangle\:-\:$основная пара для терма $M$, если 
		 \begin{enumerate}
			\item $\Gamma\:\vdash M:\tau$
			\item Если $\Gamma'\:\vdash M:\tau'$, то существует $S:\:S(\Gamma)\:\subset\:\Gamma'$
		 \end{enumerate}
		 \begin{example}
		\end{example}
		 	Рассмотрим терм: $\lambda f\:\lambda x.\:f(f(x))$, построим и пронумеруем его дерево разбора:\par
\begin{tikzpicture}
	\node [fill=none, draw=none] (A) at (4,0) {$\lambda f\:\lambda x.\:f(f(x))\enspace{\color{blue} (7)}$};
	\node [fill=none, draw=none] (B) at (7.5,-1) {$\lambda x.\:f(f(x))\enspace{\color{blue} (6)}$};
	\node [fill=none, draw=none] (C) at (2,-1) {$\lambda f$};
	\node [fill=none, draw=none] (D) at (10.5,-2) {$f(f(x))\enspace{\color{blue} (5)}$};
	\node [fill=none, draw=none] (E) at (5.5,-2) {$\lambda x$};
	\node [fill=none, draw=none] (F) at (13.5,-3) {$f(x)\enspace{\color{blue} (4)}$};
	\node [fill=none, draw=none] (G) at (8.5,-3) {$f\enspace{\color{blue} (3)}$};
	\node [fill=none, draw=none] (H) at (11.5,-4) {$f\enspace{\color{blue} (2)}$};
	\node [fill=none, draw=none] (I) at (16.5,-4) {$x\enspace{\color{blue} (1)}$};
	\path [->] (A) edge node[left] {} (B);
	\path [->] (A) edge node[left] {} (C);	
	\path [->] (B) edge node[left] {} (D);
	\path [->] (B) edge node[left] {} (E);
	\path [->] (D) edge node[left] {} (F);
	\path [->] (D) edge node[left] {} (G);
	\path [->] (F) edge node[left] {} (H);	
	\path [->] (F) edge node[left] {} (I);
\end{tikzpicture}\par


\begin{enumerate}
	\item $\big\langle E_1, \tau_1\big\rangle=\big \langle \{\},\:\alpha_x \big \rangle$
	\item $\big\langle E_2, \tau_2\big\rangle=\big \langle \{\},\:\alpha_f \big \rangle$
	\item $\big\langle E_3, \tau_3\big\rangle=\big \langle \{\},\:\alpha_f \big \rangle$
	\item $\big\langle E_4, \tau_4\big\rangle=\big \langle \{\alpha_f=\:\rightarrow(\alpha_x\:\alpha_1)\}, \alpha_1 \big \rangle$
	\item $\big\langle E_5, \tau_5\big\rangle=\big \langle \begin{Bmatrix}
	\alpha_f=\:\rightarrow(\alpha_x\:\alpha_1)\\
	\alpha_f=\:\rightarrow(\alpha_1\:\alpha_2)
\end{Bmatrix},\:\alpha_2 \big \rangle$
	\item $\big\langle E_6, \tau_6\big\rangle=\big \langle \begin{Bmatrix}\alpha_f=\:\rightarrow(\alpha_x\:\alpha_1)\\ \alpha_f=\:\rightarrow(\alpha_1\:\alpha_2)\end{Bmatrix},\:\alpha_x\rightarrow\alpha_2 \big \rangle$
	\item $\big\langle E_7, \tau_7\big\rangle=\big \langle \begin{Bmatrix}\alpha_f=\:\rightarrow(\alpha_x\alpha_1)\\ \alpha_f=\:\rightarrow(\alpha_1\alpha_2)\end{Bmatrix},\:\alpha_f\rightarrow(\alpha_x\rightarrow\alpha_2) \big \rangle$
\end{enumerate}
	$E=\begin{Bmatrix}\alpha_f=\:\rightarrow\:(\alpha_x\:\alpha_1)\\ \alpha_f=\:\rightarrow\:(\alpha_1\:\alpha_2)\end{Bmatrix}$, решим полученную систему:\par 
	\begin{enumerate}
		\item Решим систему:\par 
			\begin{enumerate}
			\item \[\begin{cases}
				\alpha_f=\rightarrow(\alpha_x\:\alpha_1)&\\
				\alpha_f=\rightarrow(\alpha_1\:\alpha_2)&\\
			\end{cases}\]
			\item \[
				\begin{cases}
				\rightarrow(\alpha_1\:\alpha_2)=\rightarrow(\alpha_x\:\alpha_1)
				\end{cases}\]
			\item \[
				\begin{cases}
				\alpha_1=\alpha_x&\\
				\alpha_2=\alpha_1
				\end{cases}\]
			\item \[
				\begin{cases}
				\alpha_1=\alpha_x&\\
				\alpha_2=\alpha_x
				\end{cases}\]
\end{enumerate}					
		\item  Получим \[S=\begin{cases}
						\alpha_f=\rightarrow(\alpha_x\:\alpha_1)&\\
						\alpha_1=\alpha_x&\\
						\alpha_2=\alpha_x&\\
				\end{cases}\]
		\item $\Gamma=\{\}$, так как в заданной формуле нет свободных переменных
		\item Тип  терма $\lambda\:f\lambda\:x.f(f(x))$ является результатом подстановки\par $S(\rightarrow\:\alpha_f\:(\alpha_x\rightarrow\alpha_2))$, получаем $\tau=(\alpha_x\rightarrow\alpha_x)\rightarrow(\alpha_x\rightarrow\alpha_x)$
	\end{enumerate}
	\subsection{Сильная и слабая нормализации}
	\begin{definition}Если существует последовательность редукций, приводящая терм $M$ в нормальную форму, то $M-$слабо нормализуем. (Т.е. при редуцировании терма $M$ мы можем не прийти в н.ф.)\end{definition} 
	\begin{definition}Если не существует бесконечной последовательности редукций терма $M$, то терм $M-$ сильно нормализуем.\end{definition} 
	 \begin{statement}
	 \end{statement}
	 \begin{enumerate}
	 	\item $KI\Omega-$ слабо нормализуема
	 		\begin{example}\end{example}
	 		Перепишем $KI\Omega$ как ${\color{red}(}(\lambda x\:\lambda y.\:x)(\lambda x.\:x){\color{red})}{\color{blue}(}((\lambda x.\:x\:x)(\lambda x.\:x\:x)){\color{blue})}$, очевидно, что этот терм можно средуцировать двумя разными способами:\\
	 		\begin{enumerate}
	 			\item Сначала редуцируем красную скобку 
	 				\begin{enumerate}
	 					\item ${\color{red}(}(\lambda x\:\lambda y.\:x)(\lambda x.\:x){\color{red})}{\color{blue}(}((\lambda x.\:x\:x)(\lambda x.\:x\:x)){\color{blue})}$
	 					\item ${\color{red}(}(\lambda y.\:(\lambda x.\:x)){\color{red})}{\color{blue}(}((\lambda x.\:x\:x)(\lambda x.\:x\:x)){\color{blue})}$
	 					\item $(\lambda x.\:x)$
	 				\end{enumerate}
	 				Видно, что в этом случае количество шагов конечно.
	 			\item Редуцируем синюю скобку. Очевидно, что комбинатор $\Omega$ не имеет нормальной формы, тогда понятно, что в этом случае терм $KI\Omega$ никогда не средуцируется в нормальную форму.
	 		\end{enumerate}
	 	\item $\Omega-$ не нормализуема
	 	\item $II-$ сильно нормализуема
	 \end{enumerate}
	 \begin{lemma}Сильная нормализация влечет слабую.
	 \end{lemma}
	 \subsection{Выразимость комбинаторов}
     \begin{statement}Для любого $\lambda$-выражение без свободных переменных существует $\beta$-эквивалентное ему выражение, записываемое только с помощью комбинаторов $S$ и $K$, где\end{statement}
	 $S\:=\:\lambda x\:\lambda y\:\lambda z.\:(x\:z)\:(y\:z)\: : \:(a\:\rightarrow\:b\:\rightarrow\:c)\:\rightarrow\:(a\:\rightarrow\:b)\:\rightarrow\:a\:\rightarrow\:c$\par 
	 $K=\lambda x\:\lambda y.\:x\: : a\:\rightarrow\:b\:\rightarrow\:a$\par 
	 \begin{statement}Комбинаторы $S$ и $K$ являются аксиомами в ИИВ\end{statement}
	\begin{statement}Соотношение комбинаторов с $\lambda$ исчислением:\end{statement}
	\begin{enumerate}
		\item $T(x)=x$
		\item $T(P\:Q)=T(P)\:T(Q)$
		\item $T(\lambda\:x.P)=K(T(P)),\enspace x\not\in FV(P)$
		\item $T(\lambda\:x.x)=I$
		\item $T(\lambda\:x\lambda\:y.P)=T(\lambda\:x.\:T(\lambda\:y.P))$
		\item $T(\lambda\:x.P\:Q)=S\:\:T(\lambda\:x.P)\:T(\lambda\:x.Q)$
	\end{enumerate}
	\begin{statement}Альтернативный базис:\end{statement}
	 \begin{enumerate}
		\item $B=\lambda x\:\lambda y\:\lambda z.\:x\:(y\:z)\: : \:(a\:\rightarrow\:b)\:\rightarrow\:(c\:\rightarrow\:a)\:\rightarrow\:c\:\rightarrow\:b$
		\item $C=\lambda x\:\lambda y\:\lambda z.\:((x\:z)\:y)\: : \:(a\:\rightarrow\:b\:\rightarrow\:c)\:\rightarrow\:b\:\rightarrow\:a\:\rightarrow\:c$
		\item $W=\lambda x\:\lambda y.\:((x\:y)\:y)\: : \: (a\:\rightarrow\:a\:\rightarrow\:b)\:\rightarrow\:a\:\rightarrow\:b$
		\\\\\\
	 \end{enumerate}

\section{Лекция 7}
	\subsection{Импликационный фрагмент ИИП второго порядка}
 	\begin{center}
 		\begin{definition}
 			Назовем \textit{\underline{грамматикой ИИП второго порядка}} конструкцию вида: 
 		\end{definition}
 	\textbf{A} ::= 
 	(\textbf{A}) |
 	\textbf{p} |
 	\textbf{A} $\rightarrow$ \textbf{A} |
 	$\forall$\textbf{p}.\textbf{A} 
 	
 	\end{center}
 
 	В этой системе все остальные связки могут быть выражены через основные 4, представленные выше. Например, $\perp$ представима в следующем виде
 	\begin{center}
 		{\textbf{\textsl\textit{$\forall$p.p}}} \\
 	\end{center}
 	
 	
 	 Также добавим два новых правила вывода для квантора существования и два для квантора всеобщности к уже существующим в ИИВ: \\ \\
 	
	Для квантора всеобщности: \\ 
 	
    \[ \dfrac{\Gamma\vdash\phi}{\Gamma\vdash\forall p.\phi} (p\notin FV(\Gamma)) \qquad
        \dfrac{\Gamma\vdash\forall p.\phi}{\Gamma\vdash\phi[p:=\Theta]} \]
 	
 	 И два для квантора существования: \\
 	
    \[ \dfrac{\Gamma\vdash\phi[p:= \psi]}{\Gamma\vdash\exists p.\phi}\qquad
        \dfrac{\Gamma\vdash\exists p.\phi\quad\Gamma, \phi\vdash\psi}{\Gamma\vdash\psi} (p\notin FV(\Gamma, \psi)) \]
 	
 	
 	\begin{definition}
 		Грамматику ИИП второго порядка с приведенными выше правилами вывода назовем Импликационным фрагментом ИИВ второго порядка\\ 
 	\end{definition}
	{С помощью этих правил вывода можно доказать, что \textbf{${\perp = \forall p.p}$}
		Действительно, воспользовавшись вторым правилом вывода квантора всеобщности для этого выражения, мы можем вывести любое другое выражение.}
 	
 	С помощью правил вывода также можно доказать, что \\
 	$\phi\&\psi\equiv\forall a. ((\phi\rightarrow\psi\rightarrow a)\rightarrow a)$\\
 	$\phi\vee\psi\equiv\forall a. ((\phi\rightarrow a)\rightarrow(\psi\rightarrow a)\rightarrow a)$
 	
 	Докажем например, что
 	\begin{center}
 		 $\dfrac{\Gamma\vdash\perp}{\Gamma\vdash \phi}$
 	\end{center}
 	Воспользуемся вторым правилом вывода для квантора всеобщности
 	\begin{center}
		$\dfrac{\Gamma\vdash\forall\alpha.\alpha}{\Gamma\vdash\alpha[\alpha:=\phi]} $
 	\end{center}
 	
 	\subsection{Теория Моделей}
	Добавим к нашему исчислению модель. Напомню, что модель это функция которая сопоставляет некому терму элемент из множества истинностных значений. В нашем случае мы будем сопоставлять высказываниям элементы из множества $\llbracket\text{\textbf{И,Л}}\rrbracket$ по следующим правилам: \\

\begin{center}
	 	$\llbracket p\rrbracket=p$, т. е. $\llbracket p\rrbracket^{p = x} = x$ \\
\end{center}
 	
 	
 \begin{center}
 		\begin{equation*}
 		\llbracket p\rightarrow Q\rrbracket = 
 		\begin{cases}
 			\text{Л}, \llbracket p\rrbracket = \text{И}, \llbracket Q\rrbracket = \text{Л} \\
 			\text{И}, \text{иначе}
 		\end{cases}
 	\end{equation*}
 \end{center}
 	
 	
 	\begin{equation*}
 		\llbracket\forall p.Q\rrbracket = 
 		\begin{cases}
 			\text{И}, \llbracket Q\rrbracket^{p=\text{л, и}} = \text{И} \\
 			\text{Л}, \text{иначе}
 		\end{cases}
 	\end{equation*}
 	Эта модель корректна, но не полна.
 	
 	\subsection{Система F}
 	
\begin{definition}
	 	Под типом в системе F будем понимать следующее

 	
 	\begin{equation*}
 	\tau =
 	\begin{cases}
 	\alpha,\beta,\gamma ...\quad(\text{атомарные типы}) \\
 	\tau\rightarrow\tau \\
 	\forall\alpha.\tau\qquad(\alpha\text{ - переменная})
 	\end{cases}
 	\end{equation*}
 \end{definition}

 \begin{definition}
 		Введем определение грамматики в системе F:
 	\begin{center}
 		$\Lambda$ ::= x | $\lambda x^{\tau}.\Lambda$ | $\Lambda\Lambda$ | ($\Lambda$) | $\Lambda\alpha.\Lambda$ | $\Lambda\tau$ 
 	\end{center}
 \end{definition}
 	
	где $\Lambda\alpha.\Lambda$ --- типовая абстракция, явное указание того, что вместо каких-то типов мы можем подставить любые выражения, а $\Lambda\tau$ --- это применение типа. \\
 	
 	
 	Так, пример типовой абстракции это: 
    \begin{minted}{cpp}
template<typename T>
class W {
    T x;
}
    \end{minted}
 	
	Типовая аппликация --- это объявление переменной класса с каким-то типом
 	
    \begin{minted}{cpp}
W<int> w_test;
    \end{minted}
 	
 	\begin{theorem}
 		Изоморфизм Карри - Ховарда:
    \begin{center}
 		$\Gamma\vdash_F M:\tau\Leftrightarrow |\Gamma|\vdash_{\forall, \rightarrow}\tau$ 
    \end{center}
 	
 	\end{theorem}
 	
 	
 	
 	В системе F определены следующие правила вывода: \\ \\
  	\noindent
 	{$\dfrac{}{\Gamma,x:\tau\vdash x:\tau}\qquad\qquad$} 
 	{$\dfrac{\Gamma\vdash M:\sigma\rightarrow\tau\qquad\Gamma\vdash N:\sigma}{\Gamma\vdash M N:\tau}$}\\  \\ \\
 	{$\dfrac{\Gamma,x:\tau\vdash M:\sigma}{\Gamma\vdash\lambda x^{\tau}.M:\tau\rightarrow\sigma}\quad(x\notin FV(\Gamma))$}\\ \\ \\
 	{$\dfrac{\Gamma\vdash M:\sigma}{\Gamma\vdash\Lambda\alpha.M:\forall\alpha.\sigma}\quad(\alpha\notin FV(\Gamma))\qquad$}
 	 $\dfrac{\Gamma\vdash M:\forall\alpha.\sigma}{\Gamma\vdash M\tau:\sigma[\alpha:=\tau]}$
 	\\
 	
 	$\emph{Приведем пример}$. Покажем как выглядит в системе F левая проекция.
 	В просто типизированном $\lambda$ - исчислении $\pi_1$ имеет тип $\alpha\&\beta\rightarrow\alpha$. В системе F явно указывается, что элементы пары могут быть любыми и пишется соответственно $\forall\alpha.\forall\beta.\alpha\&\beta\rightarrow\alpha$. Само выражение для проекции также изменится и будет иметь вид  $\pi_1=\Lambda\alpha.\Lambda\beta.\lambda p^{\alpha\&\beta}.p\alpha\rm{T}$
 	
 	Давайте определим еще несколько понятий из простого $\lambda$-исчисления. \\
    \emph{Начнем с $\beta$-редукции:}
    \begin{enumerate}
        \item Типовая $\beta$-редукция: $(\Lambda\alpha.M^{\sigma})\tau\rightarrow_\beta M[\alpha:= \tau]:\sigma[\alpha:= \tau]$
        \item Классическая $\beta$-редукция: $(\lambda x^{\sigma}.M)^{\sigma\rightarrow\tau}X\rightarrow_\beta M[x:=X]:\tau$ 
    \end{enumerate}
 	\emph{Выразим еще несколько функций} \\
 	
    \begin{enumerate}
        \item Не бывает М:$\perp$
        \item Рассмотрим пару <P, Q> ::= $\Lambda\alpha.\lambda z^{\tau\rightarrow\sigma\rightarrow\alpha}.z P Q$ \\
            Проекторы мы рассмотрели ранее.
        \item $in_L(M^{\tau}) ::= \Lambda\alpha.\lambda u^{\tau\rightarrow\alpha}.\lambda\omega^{\sigma\rightarrow\alpha}.u M$\\
            $ in_R(M^{\sigma}) ::= \Lambda\alpha.\lambda u^{\tau\rightarrow\alpha}.\lambda\omega^{\sigma\rightarrow\alpha}.u M$\\
    \end{enumerate}
 	

 	
    \begin{enumerate}
        \item Теорема Чёрча-Россера и прочие теоремы, доказуемые в строго-типизированном лямбда-исчислении, доказуемы и в системе F
 	    \item $\lambda_{(\forall, \rightarrow)}$ Система F сильно нормализуема
 	    \item Y комбинатор не типизируем
	    \item Исчисление неразрешимое, но не противоречивое
    \end{enumerate}

\section{Лекция 8}
	\subsection{Ранг типа}
	\begin{definition}
		 Введем определение. Под {рангом типа} мы будем понимать число, получаемое по следующим правилам: \\
	    Rn(x) --- множество всех типов x\\
	    Rn(0) --- все типы без кванторов\\
	    Rn(x+1) = Rn(x) | Rn(x) $\rightarrow$ Rn(x+1) | $\forall\alpha.Rn(x+1)$
	\end{definition}
	 
	\textbf{Примеры}
	 1. $ \alpha\in $Rn(0) \\
	 2. $ \forall\alpha.\alpha \in$Rn(1)\\
	 3. $ (\forall\alpha.\alpha)\rightarrow(\forall\beta.\beta) \in$Rn(2), так как каждый тип вида $ \forall\alpha.\alpha \in$Rn(1), то по третьему правилу весь тип $ \in $Rn(2) \\

	\begin{definition}
		Тип с поверхностными кванторами --- это любой тип вида $ \forall\alpha.\tau $, где в $ \tau $ отсутствуют кванторы. Очевидно, что любой такой тип $ \in $Rn(1). Действительно, тип внутри квантора точно имеет ранг 0. Навешивание одного или нескольких кванторов всеобщности увеличит его ранг на единицу.
	\end{definition}
	 
	 \subsection{Типовая схема}
	 Возьмем только типы с поверхностными кванторами(из Rn(1)). \\
	 Также можно превратить любую формулу из Rn(1) в формулу с поверхностными кванторами. \\
	 Например:\\ 
	 $ \beta\rightarrow\forall\alpha.\alpha\equiv\forall\alpha.(\beta\rightarrow\alpha) $
	 \\
	 
	 \begin{definition}
	 	{Типовой схемой} назовем выражение вида: 

\begin{center}
		 $ \sigma\equiv\forall\alpha_1.\forall\alpha_2.....\forall\alpha_n.t $ где t$ \in $Rn(0)
\end{center}
	 	 \end{definition}
 	 
	 Также будем считать, что $ \sigma_1 <= \sigma_2 $\\
	 ($\sigma_1$ является спецификацией $\sigma_2$) если: \\\\
	 $\sigma_2\equiv\forall\alpha_1...\forall\alpha.\tau_1$\\
	 $\sigma_1\equiv\forall\beta_1...\beta_n.\tau_1[\alpha_1:=\Theta_1]...[\alpha_n:=\Theta_n]  $
	 
	 \noindent Например:\\
	 $ \forall\beta_1.\forall\beta_2.(\beta_1\rightarrow\beta_2)\rightarrow(\beta_1\rightarrow\beta_2) $\\
	 является спецификацией $ \forall\alpha.\alpha\rightarrow\alpha $\\
	 \subsection{Экзистенциальные типы}
	 1) $\dfrac{\Gamma\vdash\phi[\alpha:=\theta]}{\Gamma\vdash\exists\alpha.\phi}$\\ \\
	 2) $\dfrac{\Gamma\vdash\exists\alpha.\phi\qquad\Gamma,\phi\vdash\psi}{\Gamma\vdash\psi}$
	 
	  Экзистенциальные типы это способ инкапсуляции данных. Предположим, что у нас есть стек с хранилищем типа $\alpha$, у которого определены следующие операции:\\\\
	 \textbf{empty}: $\alpha$\\
	 \textbf{push}: $\alpha\&\nu\rightarrow\alpha$\\
	 \textbf{pop}: $\alpha\rightarrow\alpha\&\nu$\\
	 
	 Тогда очевидно, что тип stack$\equiv\alpha\&(\alpha\&\nu\rightarrow\alpha)\&(\alpha\rightarrow\alpha\&\nu)$.
	 Но что если мы реализовали хранилище как-то по-особенному, не меняя типов операций. Мы хотим скрыть данные о реализации, в частности о типе $\alpha$. Вместо деталей просто скажем, что существует интерфейс, удовлетворяющий такому типу:\\$\exists\alpha.\alpha\&(\alpha\&\nu\rightarrow\alpha)\&(\alpha\rightarrow\alpha\&\nu)$
	 
	 \subsection{Абстрактные типы}	 
	 Предположим, что мы захотим создать стек, в котором лежат целые числа. Рассмотрим, как тогда будет выглядеть тип созданного стека: \\
	 \textbf{stack}$\equiv\forall\nu.\exists\alpha.\alpha\&(\alpha\&\nu\rightarrow\alpha)\&(\alpha\rightarrow\alpha\&\nu)$\\
 	По аналогии с правилом удаления квантора существования, можно определить правила вывода для выражений абстрактных типов: \\

	
 	$\dfrac{\Gamma \vdash M : \varphi[\alpha := \theta]}{\Gamma\vdash (\text{pack } M, \theta \text{ to } \exists \alpha . \varphi) : \exists \alpha.\varphi}\\ \\ \\
	$ Это правило вывода позволяет скрыть реализацию стека, так как если $\alpha$ --- это тип стека, то $\alpha[\nu := \theta]$ --- его конкретная реализация, например ArrayStack, LinkedListStack и подобные \\ \\
 	 $
 	\dfrac{\Gamma \vdash M : \exists \alpha . \varphi\qquad\Gamma, x : \varphi \vdash N : \psi}{\Gamma \vdash \text{abstype } \alpha \text{ with } x:\varphi \text{ in } M \text{ is } N:\psi}
	(\alpha \notin FV(\Gamma, \psi))$
	\\ \\
	Это правило вывода соответствует виртуальному вызову стека какой-то реализации, например: 
	\begin{verbatim}
		foo(Stack s) {
			...
		}
	\end{verbatim}
	Поскольку выводимые формулы выглядят слишком громоздко, перепишем их, вспомнив, что: \\
	$\exists\alpha.\sigma\equiv\forall\beta.(\forall\alpha.\sigma\rightarrow\beta)\rightarrow\beta$\\\\
	Тогда: \\
	$	\text{\textbf{pack} } M, \theta \text{ \textbf{to} } \exists \alpha . \varphi =
		\Lambda \beta . \lambda x^{\forall \alpha . \varphi \to \beta} . x \theta M \\
		\text{\textbf{abstype} } \alpha \text{ \textbf{with} } x:\varphi \text{ \textbf{in} } M \text{ \textbf{is} } N:\psi =
		M \psi (\Lambda \alpha . \lambda x ^ \varphi . N)
	$
	
	 \subsection{Типовая система Хиндли-Милнера}
	   
	 Начнем с определения типа. \underline{Тип в системе Хиндли-Милнера}: \\ \\
	 Монотип --- выражение в грамматике вида $\tau::=\alpha|\tau\rightarrow\tau|(\tau)$\\
	 Политип --- выражение в грамматике вида $\sigma::=\tau|\forall\alpha.\sigma$\\
	 
	 \noindent Поэтому типы вида $\alpha\rightarrow\forall\beta.\beta$ - некорректны в системе ХМ\\
	 
	 \noindent Грамматика в системе Хиндли-Милнера имеет вид:\\
	 
	 \begin{center}
	 $\Lambda::=x | \lambda x.\Lambda | \Lambda\Lambda | (\Lambda) | \text{let x = }\Lambda \text{ in }\Lambda $ 
	 \end{center}
 		
 	 \noindent Обозначим контекст $\Gamma$ без типа x как $\Gamma_x$\\
 	 В новой системе получаем следующие правила вывода: \\ \\
 	 
 	 \noindent 1. Тавтология $\dfrac{}{\Gamma\vdash x.\phi}$ \\\\
 	
 	 \noindent 2. Уточнение $\dfrac{\Gamma\vdash e:\sigma\qquad\sigma' <= \sigma}{\Gamma\vdash e:\sigma'}$ \\\\
 	 
 	 \noindent 3. Обобщение $\dfrac{\Gamma\vdash e:\sigma}{\Gamma\vdash e:\forall\alpha.\sigma}$ \\ \\
 	 \noindent 4. Абстракция $\dfrac{\Gamma_x, x:\tau'\vdash e:\tau}{\Gamma\vdash \lambda x.e:\tau'\rightarrow\tau}$ \\ \\
 	 \noindent 5. Применение $\dfrac{\Gamma\vdash e:\tau'\rightarrow\tau\qquad\Gamma\vdash e':\tau'}{\Gamma\vdash e\ e':\tau}$ \\ \\
 	 \noindent 6. Let $\dfrac{\Gamma\vdash e:\sigma\qquad\Gamma_x,x:\sigma\vdash e':\tau}{\Gamma\vdash\text{ \textbf{let} } x = e \textbf{ in }e':\tau}$ \\
 	 
 	 Хотя в системе Хиндли-Милнера (как и во всех рассматриваемых нами типовых системах) нельзя типизировать $\mathcal{Y} $-комбинатор,
 	 можно добавить его, расширив язык.
 	 Давайте определим его как $\mathcal{Y} f = f \left(\mathcal{Y} f\right)$.
 	 Какой у него должен быть тип? Пусть $\mathcal{Y}$ принимает $f$ типа $\alpha$, и возвращает нечто типа $\beta$,
 	 то есть $\mathcal{Y}: \alpha\to\beta$.
 	 Функция $f$ должна принимать то же, что возвращает $\mathcal{Y}$, так как результат $\mathcal{Y}$ передаётся в $f$,
 	 и возвращать она должна то же, что возвращает $\mathcal{Y}$, так как тип выражений с обеих сторон равенства должен быть одинаковый,
 	 то есть $f : \beta\to\beta$
 	 Кроме того, $\alpha$ и тип $f$ это одно и то же, $\alpha=\beta\to\beta$.
 	 После подстановки и заключения свободной переменной под квантор получаем $\mathcal{Y} : \forall\beta.(\beta\to\beta)\to\beta$.
 	 
 	 Через такой $\mathcal{Y}$ можно определять рекурсивные функции, и они будут типизироваться.
 	 
 
 	 


\section{Лекция 9}

\newcommand{\idr}[1]{\mintinline{idris}{#1}}

\begin{definition}[Ранг типа]
$R(x)$ --- все типы ранга $x$.
\begin{itemize}
    \item $R(0)$ --- все типы без кванторов
    \item $R(x + 1) = R(x)\ |\ R(x) \rightarrow R(x + 1)\ |\ \forall \alpha.R(x + 1)$
\end{itemize}
\end{definition}

Например:
\begin{paracol}{2}
\begin{itemize}
    \item $\alpha \in R(0)$
    \item $\forall \alpha.\alpha \in R(1)$
    \item $(\forall \alpha.\alpha) \rightarrow (\forall b.b) \in R(2)$
    \item $((\forall \alpha.\alpha) \rightarrow (\forall b.b)) \rightarrow b \in R(3)$
\end{itemize}
\switchcolumn
Тут видно, если выражение слева от знака импликации имеет ранг $n$, то все выражение будет иметь ранг $\geq (n + 1)$.
\end{paracol}

\textbf{Утверждение}: Пусть $x$ --- выражение только с поверхностными кванторами, тогда $x \in R(1)$.

\begin{definition}[Типовая схема]\

$\sigma ::= \forall \alpha_1. \forall \alpha_2. \dots \forall \alpha_n. \tau$, где $\tau \in R(0)$ и, следовательно, $\sigma \in R(1)$.

\end{definition}

\begin{definition}[Частный случай (специализация) типовой схемы]\

$\sigma_1, \sigma_2$ --- типовые схемы

$\sigma_2$ --- частный случай $\sigma_1$ (обознается как $\sigma_1 \sqsubseteq \sigma_2$), если

\begin{enumerate}
    \item $\sigma_1 =  \forall \alpha_1. \forall \alpha_2. \dots \forall \alpha_n. \tau_1$
    \item $\sigma_2 =  \forall \beta_1. \forall \beta_2. \dots \forall \beta_m. \tau_1[\alpha_i := S(\alpha_i)]$
    \item $\forall i. \beta_i \in FV(\tau_1)$
\end{enumerate}

\end{definition}

\begin{example}\

$\forall \alpha . \alpha \rightarrow \alpha
\sqsubseteq
\forall \beta_1.\forall \beta_2: (\beta_1 \rightarrow \beta_2) \rightarrow (\beta_1 \rightarrow \beta_2)$

Вполне возможно, что в ходе замены, все типы будут уточнены ($\alpha$ уточнится как $\beta_1 \rightarrow \beta_2$).
\end{example}


\subsection{Хиндли-Милнер}

\begin{enumerate}
    \item Все типы только с поверхностными кванторами ($R(1)$)
    \item $\overline{HM} ::= p\ |\ \overline{HM}\ \overline{HM}\ |\ \lambda p. \overline{HM}\ |\ let =  \overline{HM}\ in\ \overline{HM}$
\end{enumerate}

\begin{itemize}
    \item $\exists p.\phi = \forall b.(\forall p. (\phi \rightarrow b)) \rightarrow b$
    \item $\phi \rightarrow \bot \equiv \forall b. (\phi \rightarrow b)$

    \item $\vcenter{\infer[]{\Gamma \vdash \forall b.(\forall p.(\phi \rightarrow b)) \rightarrow b}{\infer{\Gamma \vdash (\forall p. (\phi \rightarrow b)) \rightarrow b}{\infer{\Gamma, \forall p. (\phi \rightarrow b) \vdash b}{\infer{\Gamma, \forall p. (\phi \rightarrow b) \vdash \phi [p := \Theta] \rightarrow b}{\Gamma, \forall p. (\phi \rightarrow b) \vdash \forall p.(\phi \rightarrow b)}}}}}$
    % \item $\Gamma, \forall p. (\phi \rightarrow b) \vdash \forall p.(\phi \rightarrow b)$
    % \item $\Gamma, \forall p. (\phi \rightarrow b) \vdash \phi [p := \Theta] \rightarrow b$
    % \item $\Gamma, \forall p. (\phi \rightarrow b) \vdash b$
    % \item $\Gamma \vdash (\forall p. (\phi \rightarrow b)) \rightarrow b$
    % \item $\Gamma \vdash \forall b.(\forall p.(\phi \rightarrow b)) \rightarrow b$
\end{itemize}

Соглашение:
\begin{itemize}
    \item $\sigma$ --- типовая схема
    \item $\tau$ --- простой тип
\end{itemize}


\begin{enumerate}
    \item $\vcenter{\infer{\Gamma, x : \sigma \vdash x : \sigma}{}}$
    \item $\vcenter{\infer{\Gamma \vdash e_0\ e_1 : \tau'}{\Gamma \vdash e_0 : \tau \rightarrow \tau' \qquad \Gamma \vdash e_1 : \tau}}$
    \item $\vcenter{\infer{\Gamma \vdash \lambda x.e : \tau \rightarrow \tau'}{\Gamma, x : \tau \vdash e : \tau'}}$
    \item $\vcenter{\infer[,\ let\ x = a\ in\ b \equiv (\lambda x.b)\ a]{\Gamma \vdash let\ x = e_0\ in\ e_1 : \tau}{\Gamma \vdash e_0 : \sigma \qquad \Gamma, x : \sigma \vdash e_1 : \tau}}$
    \item $\vcenter{\infer{\Gamma \vdash e : \sigma}{\Gamma \vdash e : \sigma' \qquad  \sigma' \sqsubseteq \sigma}}$
    \item $\vcenter{\infer[\alpha \not \in FV(\Gamma)]{\Gamma \vdash e : \forall \alpha.\sigma}{\Gamma \vdash e : \sigma}}$
\end{enumerate}


\subsection{Алгоритм вывода типов в системе Хиндли-Милнера W}
На вход подаются $\Gamma,\ M$, на выходе наиболее общая пара $(S, \tau)$
\begin{enumerate}
    \item $M = x ,\ x:\tau  \in \Gamma$ (иначе ошибка)
    \begin{itemize}
        \item Выбросить все кванторы из $\tau$
        \item Переименовать все свободные переменные в свежие \\
        Например: $\forall \alpha_1.\phi \Rightarrow \phi[\alpha_1 := \beta_1]$, где $\beta_1$ --- свежая переменная
    \end{itemize}
    $(\emptyset, \Gamma(x))$
    \item $M = \lambda n.e$
    \begin{itemize}
        \item $\tau$ --- новая типовая переменная
        \item $\Gamma' = \Gamma \setminus \{n : \_ \}$ (т.е. $\Gamma$ без переменной $n$)
        \item $\Gamma'' = \Gamma' \cup {n : \tau}$
        \item $(S',\ \tau') = W(\Gamma'', e)$
    \end{itemize}
    $(S', S'(\tau) \rightarrow \tau')$
    \item $M = P\ Q$
    \begin{itemize}
        \item $(S_1, \tau_1) = W(\Gamma, P)$
        \item $(S_2, \tau_2) = W(S_1(\Gamma), Q)$
        \item $S_3$ --- Унификация $(S_2(\tau_1), \tau_2 \rightarrow \tau)$
    \end{itemize}
    $(S_3 \circ S_2 \circ S_1, S_3(\tau))$
    \item $let\ x = P\ in\ Q$
    \begin{itemize}
        \item $(S_1, \tau_1) = W(\Gamma, P)$
        \item $\Gamma' = \Gamma$ без $x$
        \item $\Gamma'' = \Gamma' \cup \{ x : \forall \alpha_1 \dots \alpha_k. \tau_1 \}$, где $\alpha_1 \dots \alpha_k$ --- все свободные переменные в $\tau_1$
        \item $(S_2, \tau_2) = W(S_1(\Gamma''), Q)$
    \end{itemize}
    $(S_2 \circ S_1), \tau_2)$
\end{enumerate}

Надеемся, что логика второго порядка противоречива. \\

\subsection{Рекурсивные типы}

Ранее мы уже рассматривали $Y$-комбинатор, но не могли типизировать его и отказывались.
Однако в программировании хотелось бы использовать рекурсию, поэтому тут мы введем его аксиоматически.

$Y f =_\beta f (Y\ f)$

$Y : \forall \alpha . (\alpha \rightarrow \alpha) \rightarrow \alpha$ --- аксиома

И теперь, когда мы хотим написать какую-то рекурсивную функцию, скажем, на языке Ocaml, то интерпретировать ее можно будет следующим образом:

\begin{paracol}{2}
\begin{minted}[escapeinside=||,mathescape=true]{ocaml}
let rec f = expr in
    expression
\end{minted}
\switchcolumn
\begin{minted}[escapeinside=||,mathescape=true]{ocaml}
let f = Y (|$\lambda$| f. expr) in
    expression
\end{minted}
\end{paracol}

Рекурсивными могут быть не только функции, но и типы. Как, например, список из целых чисел:

\begin{minted}{ocaml}
type intList = Nil | Cons of int * intList;;
\end{minted}

На нем мы можем вызывать рекурсивные функции, например, ниже представлен фрагмент кода, позволяющий найти длину списка.

\begin{minted}{ocaml}

let rec length l = match l with
  | Nil -> 0
  | Cons (x, s) -> 1 + length s;;

let my_list = Cons(1, Cons (2, Cons (3, Nil)));;

print_int (length my_list);;  (* output: 3 *)
\end{minted}

Рассмотрим, что из себя представляет тип списка выше:


$Nil = inLeft\ O = \lambda a. \lambda b. a\ O$

$Cons = inRight\ p = \lambda a. \lambda b. b\ p$

$\lambda a. \lambda b. a\ O : \forall \gamma .(\alpha \rightarrow \gamma) \rightarrow (\beta \rightarrow \gamma) \rightarrow \gamma$

$\lambda a. \lambda b. b\ p: \forall \gamma .(\alpha \rightarrow \gamma) \rightarrow (\beta \rightarrow \gamma) \rightarrow \gamma$

$\delta = \forall \gamma .(\alpha \rightarrow \gamma) \rightarrow (\beta \rightarrow \gamma) \rightarrow \gamma$

$\lambda a. \lambda b. b\ (\lambda a. \lambda b. a\ O) : \forall \alpha. (\alpha \rightarrow \gamma) \rightarrow (\delta \rightarrow \gamma) \rightarrow \gamma$


\vspace{5mm}
Научимся задавать рекурсивные типы, а именно рассмотрим два способа решения:

\begin{enumerate}
    \item Эквирекурсивный
    \begin{minted}{ocaml}
        list = Nil | Cons a * list
    \end{minted}
    $\alpha = f(\alpha)$ --- уравнение с неподвижной точкой. Пусть $\mu \alpha. f(\alpha) = f(\mu \alpha.f(\alpha))$. Используем это в типах, а именно $f$ --- это и тип список. То есть мы по сути использовали $Y$ комбинатор, который для выражений, а для типов ввели аналогичный $\nu$.

    На практике такой подход используется и в языке программирования Java:

    \begin{minted}{java}
        class Enum <extends Enum<E>>
    \end{minted}

    Также приведем пример вывода типа $\lambda x. x\ x$ (можно вспомнить, что именно этот терм помешал нам типизировать $Y$-комбинатор в просто типизированном $\lambda$-исчислении):

    \begin{paracol}{2}
    \switchcolumn
            $\vcenter{\infer{\vdash \lambda x. x\ x : \tau \rightarrow \beta}{\infer[]{x : \tau \vdash x\ x : \beta}{x : \tau \vdash x : \tau \rightarrow \beta \qquad x : \tau \vdash x : \tau}}}$
    \switchcolumn
        Пусть $\tau = \mu \alpha. \alpha \rightarrow \beta$. Если мы раскроем $\tau$ один раз, то получим $\tau = \tau \rightarrow \beta$. Если раскроем еще раз, то получим $\tau = (\tau \rightarrow \beta) \rightarrow \beta$.
    \end{paracol}

    Ранее мы ввели $Y$-комбинатор аксиоматически, а можем ли мы его типизировать используя рекурсивные типы? Ответ: Да, можем. Напомним, что $Y = \lambda f.(\lambda x. f\ (x\ x))\ (\lambda x. f\ (x\ x))$.

    \newcommand{\scl}{:\!} % short colon
    $\vcenter{\infer[]
      {\vdash \lambda f.(\lambda x. f\ (x\ x))\ (\lambda x. f\ (x\ x)) \scl
      \forall \beta. (\beta \rightarrow \beta) \rightarrow \beta}
      {\infer[]
        {\vdash \lambda f.(\lambda x. f\ (x\ x))\ (\lambda x. f\ (x\ x)) \scl
        (\beta \rightarrow \beta) \rightarrow \beta}
        {\infer[]
          {f\scl \beta \rightarrow \beta \vdash (\lambda x. f\ (x\ x))\
          (\lambda
          x. f\ (x\ x))\scl \beta}
          {\infer[]
            {\lambda f\scl \beta \rightarrow \beta \vdash \lambda x. f\ (x\
            x)\scl
            \tau \rightarrow \beta}
            {\infer[]
              {f\scl \beta \rightarrow \beta \vdash \lambda x. f\ (x\ x)\scl
              \tau}
              {\infer[]
                {f\scl \beta \rightarrow \beta, x\scl \tau \vdash f\ (x\ x)}
                {\lambda f \scl \beta \rightarrow \beta,\ x\scl \tau \vdash
                  f\scl \beta \rightarrow \beta
                ~~~
                f : \beta \rightarrow \beta,\ x \scl \tau \vdash x\ x\scl
                \beta}}}
             \!\!\!\!\!\!\!\!\!\!\! \infer
                {\lambda f\scl \beta \rightarrow \beta \vdash \lambda x. f\
                  (x\ x)\scl \tau}
                {\text{аналогично}}}}}}$

    Загадочка: А можно ли типизировать, скажем $\lambda x : Nat. x (S x)$?
    \item Изорекурсивный

    В отличие от эквирекурсивных типов будем считать, что $\mu \alpha. f(\alpha)$ изоморфно $f(\mu \alpha.f(\alpha))$. Такой подход используется в языке программирования C.
    \begin{minted}{C}
    struct list {
        list* x;
        int a;
    }
    (*x).(*x).(*x).a
    // или, что эквивалентно
    x->x->x.a
    \end{minted}

    Можно заметить, что выше для работы со списком мы использовали специальную операцию:
    $*: list* \rightarrow list$ --- разыменование

    В изорекурсивных типах введены специальные операции для работы с этими типами, и оператор * из C как раз был примером одной из них (в частности roll):
    \begin{itemize}
        \item $Roll: Nil | Cons (a * list) \rightarrow list$
        \item $Unroll: list \rightarrow Nil | Cons (a * list)$
    \end{itemize}

    В более общем виде (введение в типовую систему):
    \begin{itemize}
        \item $roll: f(\alpha) \rightarrow \alpha$
        \item $unroll: \alpha \rightarrow f(\alpha)$
    \end{itemize}

    Можно привести еще примеры из языка C:
    \begin{itemize}
        \item $*: T* \rightarrow T$
        \item $\&: T \rightarrow T*$
        \item $T = \alpha$
        \item $T* = f(\alpha)$
    \end{itemize}

\end{enumerate}

\subsection{Зависимые типы}

Рассмотрим функцию sprintf из языка C:

    $sprintf : string \rightarrow smth \rightarrow string$

    $sprintf "\%d" : int \rightarrow string$

    $sprintf "\%f" : float \rightarrow string$

Легко видеть, что тип sprintf определяется первым аргументом. То есть тип этой функции зависит от терма --- именно такой тип и называется зависимым (\textit{англ: dependent type}).

Рассмотрим несколько иной пример, а именно список. Предположим, что мы хотим скалярно перемножить два списка:

\begin{minted}{ocaml}
let rec dot lst1 lst2 = match (lst1, lst2) with
  | ([], []) -> 0
  | (x :: xs, y :: ys) -> x * y + (dot xs ys)
;;

dot [1; 2] [3; 4] (* results in 11 *)

dot [1; 2] [3; 4; 5] (* получим ошибку *)

\end{minted}

Было бы очень здорово уметь отлавливать эту ошибку не в рантайме, а во время
компиляции программы и зависимые типы могут в этом помочь. Например в языке
Idris можно использовать \idr{Vect}:

\begin{minted}{idris}
dot : {n : Nat} -> Vect n Integer -> Vect n Integer -> Integer
dot {n = Z} [] [] = 0
dot {n = (S len)} (x :: xs) (y :: ys) = y * x + dot xs ys

let v1 = Data.Vect.fromList [1, 2, 3]
let v2 = Data.Vect.fromList [4, 5, 6]
dot v1 v2 -- results in 32

let v1 = Data.Vect.fromList [1, 2, 3, 4]
dot v1 v2 -- Type mismatch between
          --      Vect 3 Integer (Type of v2)
          -- and
          --      Vect 4 Integer (Expected type)
\end{minted}

Если подойти к типу функции \idr{dot} ближе с точки зрения теории типов, то мы
бы записали это так (о * речь пойдет в следующей главе [стоит ее воспринимать
как тип типа]):

\idr{Nat:*, Integer:*, Vect : Nat -> Integer -> *}
$\vdash$\\
$\Pi$ \idr{n:Nat} $.$ \idr{Vect n Integer -> Vect n Integer -> Integer}

\subsubsection{$\Pi$-типы и $\Sigma$-типы}

\begin{itemize}
    \item $\Pi x : \alpha . P(x)$ - эту запись можно читать как (в каком-то смысле в интуиционистском понимании): "У меня есть метод для конструирования объекта типа $P(x)$, использующий любой предоставленный $x$ типа $\alpha$". Если же смотреть на эту запись с точки зрения классической логики, то ее можно понимать как бесконечную  конъюнкцию $P(x_1)\&P(x_2)\&...$. Данная конъюнкция соответствует декартовому произведению, отсюда и название $\Pi$-типа (иногда в англоязычной литературе можно встретить \textit{dependent function type}).
    \item $\Sigma x : \alpha . P(x)$. Аналогично предыдущему пункту рассмотрим значение с интуиционистской точки зрения: "У меня есть объект $x$ типа $\alpha$, но больше ничего про него не знаю кроме того, что он обладает свойством $P(x)$". Это как раз в стиле интуиционизма, что нам приходится знать и объект $x$ и его свойство $P(x)$. Это можно представить как пару, а пара - бинарное произведение. С точки же зрения классической логики, мы можем принимать эту формулу как бесконечную дизъюнкцию $P(x_1) \vee P(x_2)\vee ...$, которая соответствует алгебраическим типам данных. (иногда в англоязычной литературе можно встретить \textit{dependent sum}).
\end{itemize}

Ранее обсуждалось, что тип может быть сопоставлен множеству его значений, как например тип uint32\_t в С++ может быть сопоставлен множеству $\{0, 1, ..., 2^{32} - 1\}$. Рассмотрим $\Pi x : \alpha . P(x)$: этому $\Pi$-типу можно сопоставить прямое произведение $B^A$ (где $A$ --- множество, сопоставленное типу $\alpha$, а $B(a)$ --- множество, сопоставленное типу $P(a)$), которое следует воспринимать, как $B^A = \prod_{a \in A} B(a) = \{ f : A \rightarrow \bigcup_{a \in A} B(a)\ |\ f(a) \in B(a), a \in A  \}$. Можно отметить, что если $B(a) = C = const$, то на любой вход $f(a) \in C$, т.е. тип значения $f(a)$ не меняется, собственно поэтому этот тип в таком случае записывают как $A \rightarrow P$. Рассмотрим $\Sigma x : \alpha . P(x)$: этому $\Sigma$-типу можно сопоставить дизъюнктное объединение $\sqcup_{a \in A} B(a) = \bigcup_{a \in A}\{(a, x) | x \in B(a)\}$, где $A$ --- множество, сопоставленное типу $\alpha$, а $B(a)$ - множество, сопоставленное типу $P(a)$. Тут также можно отметить, что если $B(a) = C = const$, то результатом дизъюнктивного объединения будет прямое произведение $A \times B$. В языке программирования Idris примером $\Sigma$-типа является зависимая пара:

\begin{minted}{idris}
data DPair : (a : Type) -> (P : a -> Type) -> Type where
  MkDPair : {P : a -> Type} -> (x : a) -> P x -> DPair a P
\end{minted}

Также есть некоторый синтаксический сахар (a : A ** P), который обозначает зависимую пару типа DPair A P, где P может содержать в себе имя a.

В документации Idris'а есть хороший пример использования: мы хотим отфильтровать вектор  (Vect) по какому-то предикату - мы не можем знать заранее длину результирующего вектора, поэтому зависимая пара выручает:

\begin{minted}{idris}
filter : (a -> Bool) -> Vect n a -> (p ** Vect p a)
filter p Nil = (_ ** [])
filter p (x :: xs) with (filter p xs)
  | ( _ ** _xs ) = if (p x) then
                    ( _ ** x :: _xs )
                  else
                    ( _ ** _xs )
\end{minted}


\section{Лекция 10}

\subsection{Введение}

Прежде мы разбирали просто типизированное лямбда-исчисление, в котором термы зависели от термов, например, терм $(F\ M)$ зависит от терма $M$. После того, как было замечено, что, скажем, $I$ может иметь разные типы, которые по сути различаются лишь аннотацией, например, $\lambda x. x : \alpha \rightarrow \alpha$, $\lambda x. x : (\alpha \rightarrow \alpha) \rightarrow (\alpha \rightarrow \alpha)$, была введена типовая абстракция, то есть термы теперь могли зависеть от типов и такая типовая система была названа System F и можно было писать $\Lambda \alpha. \lambda x : \alpha . x : \forall \alpha. \alpha \rightarrow \alpha$. То есть это было своего рода изобретением шаблонов в языке C++. Но на этом все не ограничено. System $F_w$, в которой типы могут зависеть от типов, как, например, список - алгебраический тип данных, у которого есть две альтернативы $Nil : \forall \alpha . List \alpha$ и $Cons : \forall \alpha. \alpha \rightarrow List \alpha \rightarrow \alpha$ (рекурсивные типы смотри выше). Для лучшего понимания различия системы $F$ и $F_w$ ниже представлены грамматики для типов:
\begin{itemize}
    \item $T_\rightarrow ::= \alpha\ |\ (T_\rightarrow)\ |\ T_\rightarrow \rightarrow T_\rightarrow$
    \item $T_F ::= \alpha\ |\ \forall \alpha. T_F\ |\ (T_F)\ |\ T_F \rightarrow T_F$
    \item $T_{F_w} ::= \alpha\ |\ \lambda \alpha. T_{F_w}\ |\ (T_{F_w})\ |\ T_{F_w} \rightarrow T_{F_w}\ |\ T_{F_w}\ T_{F_w} $
\end{itemize}

Ничего не мешает рассматривать типовую систему, в которой тип может зависеть от терма, как это было сделано раньше. Пусть для всех $a : \alpha$ мы можем определить тип $\beta_\alpha$ и пусть существует $b_\alpha : \beta_\alpha$. Тогда вполне обоснована запись функции $\lambda \alpha : b_\alpha$. Тип данного выражения принято записывать как $\Pi a :\alpha . \beta_\alpha$ (стоит сделать замечание, что если $\beta_\alpha$ не зависит от $\alpha$ [то есть функция константа], то вместо $\Pi a :\alpha . \beta_\alpha$ пишут $\alpha \rightarrow \beta$). Примером может быть тип вектора, длина которого зависит от натурального числа и типа (пример из языка Idris):
\begin{minted}{idris}
data Vect : (len : Nat) -> (elem : Type) -> Type where
  Nil  : Vect Z elem
 (::) : (x : elem) -> (xs : Vect len elem) -> Vect (S len) elem
\end{minted}

Теперь наша грамматика стала обширной и появилась необходимость более формально говорить о типах, т.е. ввести их в систему. Для этого был придуман род (\textit{англ: kind}), который обозначают $*$. Используя $*$ можно задавать типы типовых конструкторов.

Рассмотрим пару примеров, как используется род:

\begin{itemize}
    \item $\lambda m : \alpha.F\ m : (\alpha \rightarrow \beta) : *$
    \item $\lambda \alpha : *.I_\alpha : (\Pi \alpha : * . \alpha \rightarrow \alpha):*$
    \item $\lambda n : Nat . A^n \rightarrow B : Nat \rightarrow *$
    \item $\lambda a : *. a \rightarrow a : * \rightarrow *$
\end{itemize}

Попробуем разобраться, что же написано в примерах.

\begin{itemize}

    \item Первый пример --- это типизация привычной нам абстракции. Утверждение $a \rightarrow b : *$ значит $a \rightarrow b$ --- это тип.

    \item Во втором примере мы рассматриваем лямбда-выражение, которое принимает на вход тип и возвращает терм $I_\alpha$. Таким образом мы собираемся типизировать терм, зависящий от типа. Для этого как сказано выше мы вводим символ $\Pi$, а вот в известной нам системе F тип выражения $\lambda \alpha : *.I_\alpha$ был бы $\forall \alpha. (\alpha \rightarrow \alpha)$.

    \item В третьем пункте мы хотим сформировать утверждения для типа, зависящего от терма. Интуитивно понятно, что у такого выражения будет род $Nat \rightarrow *$. И заселять его будут конструкторы типов, которые принимают на вход число и возвращают тип, например $\lambda x : Nat. int [x]$ --- это терм, который заселяет род  $Nat \rightarrow *$

    \item В четвертом пункте мы типизируем конструктор типа, который принимает на вход тип. Действительно, его родом будет $* \rightarrow *$.

\end{itemize}

Возникает желание каким-то образом объединить все роды, и это необходимо для дальнейшей формализации происходящего. $* \rightarrow * : ?$. Что можно поставить на место вопросика? Это не тип, так как иначе бы могли записать $* \rightarrow * : *$, однако понятно, что это не так. В частности, для этого вводится понятие сорта (\textit{англ. sort}), которое можно воспринимать как тип рода и тогда $* \rightarrow * : \openbox $ и $* : \openbox$. Для любого выражения вида $A \rightarrow *$, где A --- это что угодно, верно, что оно типизируется $\openbox$. Например,

$* \rightarrow * \rightarrow * : \openbox$ - этот род очень похож на $*
\rightarrow *$, и действительно, единственное отличие заключается в количестве
аргументов нашего типового конструктора. В частности, этот род заселяет
конструктор map, $\lambda keyType : *. (\lambda valueType. map<keyType,
valueType>)$

Теперь мы ознакомились со всеми необходимыми обозначениями и неформальными определениями. Обобщая все вышесказанное, построим обобщенную типовую систему.

\subsection{Обобщенная типовая система}
\begin{itemize}

\item Сорта: \{*, \openbox\}
\begin{itemize}
    \item Выражение "$A:*$" означает, что $A$ --- тип. И тогда, если на метаязыке мы хотим сказать "Если $A$ тип, то и $A \rightarrow A$ тоже тип", то формально это выглядит как $A:* \vdash (A \rightarrow A):*$
    \item $\openbox$ - это абстракция над сортом для типов.
    \item Например:
    \begin{itemize}[leftmargin = 2cm]
        \item $5:int:*:\openbox$
        \item $[]:*\rightarrow*:\openbox$
        \item $\Lambda M.List<M>:*\rightarrow* : \openbox$
    \end{itemize}
\end{itemize}

\item $T ::= x\ |\ c\ |\ T\ T\ |\ \lambda x:T.\ T\ | \Pi x:T.\ T$

\item Аксиома:
\begin{itemize}
    \item $\vcenter{\infer{\vdash * : \openbox}{}}$
\end{itemize}

\item Правила вывода:
\begin{enumerate}
    \item $\vcenter{\infer[x \not \in \Gamma]{\Gamma, x : A \vdash x : A}{\Gamma \vdash A:S}}$
    \item $\vcenter{\infer[\text{--- правило ослабления (примерно как } \alpha \rightarrow \beta \rightarrow \alpha \text{ в И.В.)}]{\Gamma, x : C \vdash A:B}{\Gamma \vdash A:B \qquad \Gamma \vdash C:S}}$
    \item $\vcenter{\infer[\text{--- правило конверсии}]{\Gamma \vdash A:B'}{\Gamma \vdash A:B \qquad \Gamma \vdash B':S \qquad B =_\beta B'}}$
    \item $\vcenter{\infer[\text{--- правило применения}]{\Gamma \vdash (F\ a) : B[x := a]}{\Gamma \vdash F : (\Pi x:A.B) \qquad \Gamma \vdash a : A}}$
\end{enumerate}

\item Семейства правила (generic-правила)

Пусть $(s_1, s_2) \in S \subseteq \{*, \openbox\}^2$.

\begin{enumerate}
    \item $\Pi$-правило: $\vcenter{\infer[]{\Gamma \vdash (\Pi x : A.B) : s_2}{\Gamma \vdash A : s_1 \qquad \Gamma, x : A \vdash B : s_2}}$
    \item $\lambda$-правило: $\vcenter{\infer[]{\Gamma \vdash (\lambda x : A . b) : (\Pi x : A. B)}{\Gamma \vdash A:s_1 \qquad \Gamma, x : A \vdash b : B \qquad \Gamma, x : A \vdash B : s_2}}$
\end{enumerate}

\end{itemize}

В одном из примеров мы рассмотрели утверждение $\lambda \alpha : *.I_\alpha : (\Pi \alpha : * . \alpha \rightarrow \alpha):*$. Теперь мы можем до конца понять, почему $(\Pi \alpha : * . \alpha \rightarrow \alpha):*$ и что такое $\Pi$. Неформально говоря, $\Pi$-правило говорит нам о том, что выражение $(\Pi x : A.B)$ типизируется либо $*$, либо $\openbox$, а именно тем, чем является B. То есть, $(\Pi x : A.B)$ --- это либо тип конструктора типа, либо тип конструктора терма. В приведенном примере мы принимаем на вход любой тип $\alpha$  и возвращаем терм, а значит $(\Pi \alpha : * . \alpha \rightarrow \alpha):*$.

Еще пару слов про $\Pi$. Этот символ является обобщением $\rightarrow$, поэтому, во всех рассмотренных ранее родах, согласно нашей обобщенной типовой системе, можно заменить $\rightarrow$ на $\Pi$, согласно замечанию выше. Например, $* \rightarrow * = \Pi a : *. *$. Важно понимать, что подразумевается под зависимостью тела от аргумента и не путать понятия терм и тип. В $\Pi a : *. *$ тело не зависит от аргумента, потому что тело --- это просто звездочка, то есть  $\Pi a : *. *$ говорит нам просто о том, что наше выражение принимает тип и выдает тип. В то время как термы, населяющие $\Pi a : *. *$, разумеется, могут иметь тело, зависящее от аргумента, как, например, $\lambda a : *. a \rightarrow a$

\subsection{$\lambda$-куб}

В обобщенных типовых системах есть generic-правила, которые зависят от выбора $s_1$ и $s_2$ из множества сортов. Этот выбор можно проиллюстрировать в виде куба.

\begin{center}
    {\includegraphics[scale=0.5]{pic.png}}
\end{center}

Выбор правил означает следующее:
\begin{itemize}
    \item $(*,\ *)$ - позволяет записывать термы, которые зависят от термов
    \item $(\openbox,\ *)$ - позволяет записывать термы, которые зависят от типов
    \item $(*,\ \openbox)$ - позволяет записывать типы, которые зависят от термов
    \item $(\openbox,\ \openbox)$ - позволяет записывать типы, которые зависят от типов
\end{itemize}

На самом деле в данной формулировке под типом понимается не только привычный тип. Потому что для привычного типа верно $\tau : *$. Здесь же $\tau$ может типизироваться чем угодно, кроме $\openbox$. В частности $* \rightarrow *$, это значит, что например std::vector<T> тоже подходит.

Также на этом кубике можно расположить языки программирования, например:
\begin{itemize}
    \item Haskell будет располагаться на левой грани куба, недалеко от $\lambda w$
    \item Idris и Coq, очевидно, будут находиться в $\lambda C$
    \item C++ очень ограниченно приближается к $\lambda C$ (мысли вслух):
    \begin{enumerate}
        \item $(*,\ *)$ - без этого не может обойтись ни один язык программирования
        \item $(\openbox,\ *)$ - например, sizeof(type)
        \item $(*,\ \openbox)$ - например, std::array<int, 19> - тут есть ограничение на то, значение каких типов можно подставлять.
        \item $(\openbox,\ \openbox)$ - например, std::vector<int>, int*
    \end{enumerate}
\end{itemize}

\subsection{Свойства}

Для систем в $\lambda$-кубе верны следующие утверждения:
\begin{itemize}
    \item \textbf{Th. SN} \qquad \qquad \qquad \quad \quad \ Обобщенная типовая система сильно нормализуема
    \item \textbf{Th. Черча-Россера} \quad \begin{minipage}{0.6\textwidth}
\raggedright % obviates the need for explicit linebreaks
\begin{enumerate}
    \item Для любых трёх элементов $A$, $B$ и $C$, таких,
    $A \twoheadrightarrow B$ и $A \twoheadrightarrow C$ верно,
    что существует $D$, что
    $B \twoheadrightarrow D$ и $C \twoheadrightarrow D$
    \item Для любых двух элементов $A$, $B$, для которых верно $A =_\beta B$,
    существует $C$, что $A \twoheadrightarrow C$ и $B \twoheadrightarrow C$
\end{enumerate}
\end{minipage}
    \item \textbf{Th. Subject reduction} \quad \begin{minipage}{0.6\textwidth}
\raggedright % obviates the need for explicit linebreaks
    $\Gamma \vdash A : T$ и $A \twoheadrightarrow B$, тогда $\Gamma \vdash B : T$
\end{minipage}
    \item \textbf{Th. Unicity of types} \quad \ \  \begin{minipage}{0.6\textwidth}
\raggedright % obviates the need for explicit linebreaks
    $\Gamma \vdash A : T$ и $\Gamma \vdash A : T'$ тогда $T =_\beta T'$
\end{minipage}
\end{itemize}


\vspace{5mm}

Примеры:

\begin{itemize}
    \item $\lambda \omega$:
\begin{center}
    $\vdash (\lambda \alpha : * . \alpha \rightarrow \alpha) : (* \rightarrow *) : \openbox$

\vspace{5mm}

\begin{enumerate}[]
    \item \begin{center}
        $\vcenter{\infer{\vdash (* \rightarrow *) : \openbox}{\vdash * : \openbox \qquad \infer{a:* \vdash *.\openbox}{\vdash *.\openbox} }}$
    \end{center}
    \item \begin{center} $\vcenter{\infer{\vdash (\lambda \alpha : * . \alpha \rightarrow \alpha) : * \rightarrow *}{\vdash * : \openbox \qquad \infer{\alpha : * \vdash \alpha \rightarrow \alpha : x}{\alpha : * \vdash \alpha : * \qquad \alpha : *, x : \alpha \vdash \alpha : *} \qquad \infer{a:* \vdash *: \openbox}{\vdash * :\openbox} } }$
    \end{center}
\end{enumerate}
\end{center}

% \item $\lambda \rightarrow$

\end{itemize}

Notes:
\begin{itemize}
    \item $(\lambda x.x) : (A \rightarrow A)$ - implicit typing (Curry style)
    \item $I_A = \lambda x : A.x$ - explicit typing (Church style)
\end{itemize}

Рассмотрим еще примеры для улучшения понимания лямбда-куба и обобщенной типовой системы:

\begin{itemize}

	\item В системе F ($\lambda 2$) выводимо:

	\begin{enumerate}
		\item $\vdash (\lambda \alpha : * . \lambda a : \alpha . a) : (\Pi \alpha : * . (\alpha \rightarrow \alpha)) : *$

		\item $A : * \vdash (\lambda \alpha : * . \lambda a : \alpha . a) A : (A \rightarrow A)$

		\item $A : *, b : A \vdash (\lambda \alpha : * . \lambda a : \alpha . a) A b : A$

		Разумеется, здесь имеет место редукция: $(\lambda \alpha : * . \lambda a : \alpha . a) A b \rightarrow_\beta b$.

	\end{enumerate}

	\item В $\lambda \underline{w}$ выполняется

	\begin{enumerate}
		\item $\vdash (\lambda \alpha : *. \alpha \rightarrow \alpha) : * \rightarrow * : \openbox$

		\item $\beta : * \vdash (\lambda \alpha : *. \alpha \rightarrow \alpha) \beta : *$

		\item $\beta : *, x : \beta \vdash (\lambda y : \beta . x) : (\lambda \alpha : *. \alpha \rightarrow \alpha) \beta$

		\item $a : *, f : * \rightarrow * \vdash f(fa) : *$

		\item $a : * \vdash (\lambda f : * \rightarrow * . f (f a)) : (* \rightarrow *) \rightarrow * $
	\end{enumerate}

	\item В $\lambda P$ верно:

	\begin{enumerate}
	    \item $A : * \vdash (A \rightarrow *) : \openbox$

	    \item Рассмотрим тип A как множество значений типизируемых таким образом и введем $P : A \rightarrow *$
	    Тогда $A : *, P : A \rightarrow * , a : A \vdash P a : *$
	    Можно рассматривать в таком контексте P как предикат на А. Если для $a$ он возвращает населенный тип, то будем считать это за true, иначе за false. Это теоретико-множественный смысл зависимых типов.

	    Можно строить утверждения вида $(\Pi a : A. P a)$ - для любого $a$ верен предикат P.

	\end{enumerate}

    \item В $\lambda w$ можно задать конъюнкцию, как мы делали еще в системе F. $a \& b = \Pi \gamma : * . (a \rightarrow b \rightarrow \gamma) \rightarrow \gamma$

    Тогда $AND = \lambda a : *. \lambda b : *. a \& b$
    $K = \lambda a : *. \lambda b : *. \lambda x : a . \lambda y : b. x$

    $\vdash AND : * \rightarrow * \rightarrow *$

    $\vdash K : (\Pi a : *. \Pi b : *. a \rightarrow b \rightarrow a)$

    Тогда получается доказательство того, что из конъюнкции следует первый аргумент!

    $a : *, b : * \vdash (\lambda x : AND \: a b. x a (K a b)) : (AND \: a b \rightarrow a) : *$

\end{itemize}


\input{lection13}
\section{Лекция 14}

\newcommand{\ard}[1]{\mintinline{latex}{#1}}

\subsection{Индуктивные типы и равенства}

Возьмём исчисление конструкций $\lambda C$ \footnote{См. $\lambda$-куб} и дополним его базовыми конструкциями --- \textbf{индуктивными типами} и \textbf{равенством}.

\medskip
\textbf{Индуктивный тип:} это обобщение конструкций, которые можно получить с помощью индукции,
пользуясь некоторым набором базовых утверждений и индукционных переходов. В качестве примера можно рассмотреть аксиоматику Пеано.
Здесь конструкторами (выражениями, конструирующими объекты надлежащего типа) будут $0$ и $'$: если $n$ --- натуральное, то и $n'$ натуральное.

Таким образом, мы определяем новый тип, индуктивно задавая объекты, которые населяют его.

Рассмотрим реализацию натуральных чисел в языке Аренд. \footnote{\url{https://github.com/JetBrains/Arend/blob/master/lib/Prelude.ard}}
\begin{minted}[samepage]{latex}
\data Nat
    | zero
    | suc Nat
\where {
    func \infixl 6 + (x y : Nat) : Nat \elim y
        | zero => x
        | suc y => suc (x + y)
}
\end{minted}

Здесь \ard{zero} --- постулирует терм, населяющий \ard{Nat} (является базой индукции), а \ard{suc} из \ard{Nat} конструирует \ard{Nat} (таким образом, мы производим структурную индукцию).
В блоке \ard{where} задаются связанные с типом определения, а \ard{elim} представляет из себя сопоставление с образцом.

\medskip
\textbf{Равенство}. Традиционно рассматривается два типа равенств --- экстенсиональное и интенсиональное.
Интенсиональное основано на сравнении объектов по внутренней структуре, а экстенсиональное --- предполагает, что объекты неразличимы внешне.

Основными отличиями этих двух типов равенств являются разрешимость и сила. Интенсиональное --- разрешимо, но слабо, а экстенсиональное --- сильно, но неразрешимо.
Например, сравнение $0''$ и $0''$ интенсиональный подход успешно завершит, а при экстенсиональном подходе --- нам необходимо предоставить доказательство.

Кроме этого, важным отличием экстенсионального подхода является то, что в нём равенство термов по определению
не отличимо от пропозиционального равенства, которое уже доказывается внутри языка (происходит построение терма соответствующего типа).

\subsection{Пути и равенство в Arend}
В основе подхода языка Arend лежит HoTT --- Homotopy Type Theory. \footnote{The HoTT Book: \url{https://homotopytypetheory.org/book/}}

В нём произвольный тип $\alpha$ является некоторым пространством, а терм $A : \alpha$ --- точкой в нём.
Равенство же в топологии представляет из себя \textit{непрерывный} путь между двумя точками.

Введём интервальный тип \ard{I}, представляющий из себя интервал $[l, r]$.
Определим тип \ard{Path}.

\begin{minted}[samepage]{latex}
\data Path (A : I -> \Type) (a : A left) (a' : A right)
    | path (\Pi (i : I) -> A i)
\end{minted} 

Разберём это определение. Здесь \ard{A} --- это пространство, \ard{a} и \ard{a'} --- точки в нём.
Единственный конструктор является функцией, которая по левому концу пути вернёт конечную точку, а по правому концу --- начальную точку.

Теперь, с помощью путей, определим равенство.

\begin{minted}{latex}
\func \infix 1 = {A : \Type} (a a' : A) => Path (\lam _ => A) a a'
\end{minted}

Таким образом, равенство --- это функция, которая по типу и двум точкам возвращает
зависимый тип \ard{Path}, соединяющий две точки.

Следует обратить внимание на то, что в Arend запрещено на уровне компилятора выполнять
pattern matching по интервальному типу.
Иначе --- можно написать функцию, нарушающую непрерывность и впоследствии получить доказательство $0 = 1$:

\begin{minted}[samepage]{latex}
\func lr (a : I) : Nat
    | left => 0
    | right => 1
\end{minted}

\subsection{Основные функции}
\begin{itemize}
\item [\bf idp:] Вспомним определение равенства. Попробуем населить тип $0 = 0$. Это можно сделать так:

\begin{minted}{latex}
path (\lam _ => 0)
\end{minted}

На практике, необходимость доказать равенство является типичной ситуацией, и конструкция \ard{idp}
является удобным обобщением, составляющим путь по неявному аргументу \ard{a}.

\begin{minted}{latex}
\cons idp {A : \Type} {a : A} => path (\lam _ => a)
\end{minted}

\item [\bf coe:] Функция \ard{coe} позволяет <<разобрать>> равенство. Более формально --- она служит элиминатором для интервального типа.

\begin{minted}{latex}
\func coe (A : I -> \Type) (a : A left) (i : I) : A i
\end{minted}


Первый аргумент показывает, на каких типах определено равенство. Второй --- начальное значение. Третий --- интервал.
Результатом будет применение \ard{A} к \ard{i}.

С её помощью, например, можно показать, что у \ard{I} один элемент

\begin{minted}{latex}
-- Note: `left=i` is a correct identifier
\func left=i (i : I) : (left = i)
    => coe (\lam i => left = i) idp i
\end{minted}

Для доказательства \ard{left = right} можно применить эту же лемму

\begin{minted}{latex}
\func l=r : left = right => left=i right
\end{minted}

\item [\bf pmap:] Принимает функцию $f$ и тип равенства $A = B$. Возвращает тип $f(A) = f(B)$.
Пример: докажем, что если $a = b$, то $a+1 = b+1$.
\begin{minted}{latex}
\lemma example (a b : Nat) (p : a = b) 
    : (suc a = suc b) => pmap suc p
\end{minted}

\item [\bf absurd:] Позволяет получить любой тип из лжи (\ard{Empty}).
\begin{minted}{latex}
    \func absurd {A : \Type} (x : Empty) : A
\end{minted}

\item [\bf rewrite:] Принимает тип равенства $A = B$, некоторое выражение $t$, и эта функция переписывает его, подставляя $B$ вместо $A$.
Пример:
\begin{minted}{latex}
\lemma example (x y : Nat) (f : Nat -> Nat)
: f (x + y) = f (y + x)
    => rewrite (NatSemiring.+-comm {x} {y}) idp
\end{minted}

\item [\bf transport:] Эта функция является основным механизмом для работы \ard{rewrite}.
\begin{minted}{latex}
\func transport {A : \Type} (B : A -> \Type)
    {a a' : A} (p : a = a') (b : B a) : B a'
\end{minted}
\end{itemize}

\subsection{$\Sigma$- и $\Pi$-типы}
Иногда мы хотим оперировать с кортежами зависимых типов, например, если мы хотим,
чтобы одновременно удовлетворялись несколько условий. В языке Аренд сигма-тип --- это тип (зависимых) кортежей.

Покажем их использование на примере:
\begin{minted}[samepage]{latex}
\data DivisibleBy5
    | mkDiv5 (n : Nat) (\Sigma (m : Nat) (m * 5 = n))

\func ten : DivisibleBy5 => mkDiv5 10 (2, idp)
\end{minted}

Здесь, чтобы доказать, что число 10 делится нацело на 5, мы предоставили кортеж из частного $m$ и доказательства, что $m \cdot 5 = 10$.

Также, с помощью сигма-типов удобно требовать выполнение нескольких условий одновременно.
\begin{minted}{latex}
\lemma example (a b k : Nat) (p : a + b < k)
    : (\Sigma (a < k) (b < k))
\end{minted}

Чтобы доказать эту лемму, потребуется предоставить доказательства \ard{(a < k)} и \ard{(b < k)}.
Получить произвольный элемент из сигма-типа можно с помощью паттерн-матчинга.

\medskip
Вспомним реализацию путей в языке Аренд. В ней использовался пи-тип~--- функция, возвращавшая начальную или конечную точку пути.
Итак, $\pi$-тип в Arend~--- это тип зависимых функций. Такая конструкция соответствует квантору всеобщности $\forall$, так как
тип \ard{(\Pi (x : A) -> B a)} населён, когда для любого элемента \ard{a} из \ard{A} существует элемент \ard{B a}. 

Например, представим, что мы определили понятие <<делится нацело>> и хотим определить понятие простого числа $n$.
Хочется проверить, что если число делит $n$ нацело, то оно либо $1$, либо $n$. Здесь можно применить пи-типы.
\begin{minted}{latex}
\Pi (d : Nat) (k : Divisible n d) -> ((d = 1) || (d = n))
\end{minted}

\subsection{Prop, Universe}

Универсум --- это <<тип типов>>. В Arend присутствует следующая иерархия универсумов.

\begin{itemize}
    \item Все типы принадлежат универсуму 0. Например, \ard{\Type 0 => Int}
    \item Если есть функция, отображающая куда-нибудь тип, она принадлежит универсуму 1: \ard{\Type 1 => \Type -> Int}
    \item Кумулятивная последовательность --- каждый следующий элемент включает предыдущий
\end{itemize}

Такая иерархия нужна, чтобы избежать парадоксов, например, парадокса Рассела.

Концепция похожа на сорта, но при этом она включает предыдущие в иерархии. Например, \ard{\Type 100 => Int}

\medskip
Заметим, что доказательств существования \ard{Int} много --- например, $10$, $2$ или $9999$.
Давайте заведём некий набор типов, в которых всегда присутствует ровно один элемент если присутствует и назовём такой тип \textbf{собственными утверждениями}.
Чем такой тип интересен --- в нем есть утверждения, которые либо истинны, либо ложны.

Введём специальный универсум \ard{Prop}. Этот универсум состоит только из тех значений, у которых единственный элемент.

\begin{minted}{latex}
\func isProp (A : \Type) => \Pi (a a' : A) -> a = a'
\end{minted}

Такой тип может быть либо пустым, либо одноэлементным (ложь/истина).

Одно из преимуществ \ard{Prop} --- если этот тип обитаем, то мы не зависим от доказательств. Любое доказательство равно любому другому.
\begin{minted}[samepage]{latex}
\func proofIrrelevance (P : \Prop) (p q : P)
    : p = q => Path.inProp {P} p q
\end{minted}

Теперь введём понятие множества (Set). Множеством будут называться все такие элементы, у которых единственное доказательство равенства.

\begin{minted}{latex}
\func isSet (A : \Type) => \Pi (a b : A) -> isProp (a = b)
\end{minted}

Наконец, научимся делать из любого типа \ard{Prop}.

По типу \ard{a} строим тип \ard{||A||}
\begin{itemize}
    \item Если \ard{(a : A)}, то \ard{|a| : ||A||}
    \item Если \ard{(x y : A)}, то \ard{|x| = |y|}
\end{itemize}
Это называется \textbf{пропозициональным обрезанием}. В Аренде его можно сделать с помощью ключевого слова \ard{\truncated}.

Например, с помощью этой конструкции можно определить понятие <<существует>>:
\begin{minted}[samepage]{latex}
\truncated \data Exists (A : \Type) (B : A -> \Type) : \Prop
  | mkExists (a : A) (B a)
\end{minted}

Здесь тип \ard{Exists} определяет существование такого \ard{a : A}, что \ard{B a}.


\end{document}
